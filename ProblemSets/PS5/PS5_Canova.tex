\documentclass{article}

% Language setting
% Replace `english' with e.g. `spanish' to change the document language
\usepackage[english]{babel}

% Set page size and margins
% Replace `letterpaper' with `a4paper' for UK/EU standard size
\usepackage[letterpaper,top=2cm,bottom=2cm,left=3cm,right=3cm,marginparwidth=1.75cm]{geometry}

% Useful packages
\usepackage{amsmath}
\usepackage{graphicx}
\usepackage[colorlinks=true, allcolors=blue]{hyperref}

\title{PS5}
\author{Justin Canova}

\begin{document}
\maketitle



\section{Question 3 (No API)}

The data that I am using is a test for a longitudinal dataset a member of my faculty wants to build to write a paper on the evolution of organizational structure in makerspaces.  Once I get the R code fine tuned, I will select all data in the from the Dallas Makerspace to run content analysis to understand how the evolution of membership and committees evolved.  I used Claude to help me figure out the code for this first iteration.

\section{Question 4 (with API)}
For the API section, I was unable to find a meaningful API that would further my research or that of my faculty.  As such, I decided to experiment with Claude generated code and the data.gov census API calls.  I decided to use the census variables for the department of commerce to see if I could find any variables that I could use.  In their present state, I find the variables to me without value.  Perhaps I can keep experimenting and combine multiple API calls from different sources to produce a meaningful dataset.

\end{document}