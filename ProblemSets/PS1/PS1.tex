\documentclass{article}

% Language setting
% Replace `english' with e.g. `spanish' to change the document language
\usepackage[english]{babel}

% Set page size and margins
% Replace `letterpaper' with `a4paper' for UK/EU standard size
\usepackage[letterpaper,top=2cm,bottom=2cm,left=3cm,right=3cm,marginparwidth=1.75cm]{geometry}

% Useful packages
\usepackage{amsmath}
\usepackage{graphicx}
\usepackage[colorlinks=true, allcolors=blue]{hyperref}

\title{Problem Set 1}
\author{Justin Canova}

\begin{document}
\maketitle

\section{Introduction}

I find myself interested in economics because I like understanding how things work and interact together.  Economics allows for interactions, but more importantly it allows for contextualization.  Contextualization is essential, in my mind, for a phenomenon to be truly understood.  Data science is the way to take the data that comes from contextualization and makes it manageable to the point where insights can be obtained.  I am particularly interested in this class because I want to better learn how to project manage in small and large datasets, collaborate efficiently, and gather data through mean such as web scrapping to make use of the massive amounts of data that exist.  I know a little about some of the areas to be discussed in this class.  I am very excited to learn and apply proper means and methods rather than the cobbled skill set I currently possess.  The project that I am interested in pursuing in this class is a study on valuations of creative works.  I want to understand what sets successful creatives apart from the traditional "starving artist" stereotype.  I want to track art auctions, artists, how prolific they are, any monetization efforts beyond the paintings.  I also want to collect data on current Museum of Modern Art exhibits and track those artists to understand how they compare to historical popularization of art.  I am very open to any suggestions on this project.  I am hoping to gain the skills from this class to successfully gather the large volumes of primary data for this project, manage it efficiently, and turn it into a journal submission.  It is my intention after graduation to be placed at an R1 institutation as a research-based, tenure-track faculty member so I can gain further experience and practice being a researcher.
\section{Equation}
\[ a^2 + b^2 = c^2 \]

\end{document}