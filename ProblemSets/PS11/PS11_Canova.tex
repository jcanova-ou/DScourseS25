\documentclass{article}

% Language setting
% Replace `english' with e.g. `spanish' to change the document language
\usepackage[english]{babel}

% Set page size and margins
% Replace `letterpaper' with `a4paper' for UK/EU standard size
\usepackage[letterpaper,top=2cm,bottom=2cm,left=3cm,right=3cm,marginparwidth=1.75cm]{geometry}

% Useful packages
\usepackage{amsmath}
\usepackage{graphicx}
\usepackage[colorlinks=true, allcolors=blue]{hyperref}

\title{LIVING IN LIMBO: LONG-TERM PERSISTENCE IN HYBRID ENTREPRENEURSHIP}
\author{Justin Canova}

\begin{document}
\maketitle

\begin{abstract}
250 word summary.
\end{abstract}

\section{Introduction}

Summary of hybrid entreprneurship

However and purpose of the paper paragraph

Contributions paragraph

\section{Literature Review}

\subsection{Hybrid Entrepreneurship}
Hybrid entrepreneurship refers to the incremental transition to self-employment by maintaining wage employment during the transition \cite{folta2010hybrid}.  Hybrid entrepreneurial activities serve as a risk mitigation strategy for full-time entrepreneurial entry because hybrid entrepreneurs are able to experiment with and gain experience with their venture before having to commit to it full time \cite{raffiee2014should}.  Hybrid entrepreneurship is an umbrella term, comprising of many types of value creation arrangements (as seen in Table 1) such as independent contractors \cite{folta2024call}.  To a certain extent we see hybrid entrepreneurship in gig work because independent contractors offer their services through a digital intermediary \cite{burtch2018can}.  We also see hybrid entrepreneurship represented as side hustles.  Side hustles are often secondary employment \cite{sessions2021do}, but can lead to entrepreneurial opportunities given the growth of the gig economy.  Gig work can facilitate employers and labor markets to give away the liabilities of employment to outsourced labor alternatives \cite{rogiers2024end}.  
--------------------
Insert Table 1 Here
--------------------
Further terms such as part-time entrepreneurship find embrace under the hybrid entrepreneurship umbrella because working part-time as an entrepreneur does not preclude full-time wage employment too \cite{petrova2012part}.  There are more fringe terms such as moonlighting that have been implied as hybrid entrepreneurship rather than as secondary employment because moonlighting operates in parallel with primary employment (e.g. \cite{maritz2023hybrid}).  We acknowledge the various terms and blending that has been done in the past with secondary employment can be construed as hybrid entrepreneurship. Given such diversity of terms, our study strives to remain true to the definition from Folta et al. (2010) of full-time wage employment and entrepreneurship. Each of these manifestations adds to and enhances the hybrid entrepreneurship construct by acknowledging the role of the increasing flexibility brought on by the digital economy.    

In more than a decade since Folta et al. (2010) clarified the phenomenon of hybrid entrepreneurship, significant changes in the geopolitical and business landscapes have emerged.  The rapid expansion of gig work that began in the early 2010s continues to grow and evolve with the expansion of virtual work \cite{nyberg2021people} and the maker movement \cite{browder2019emergence}. The expansion of gig work and virtual work requires a reexamination of existing boundaries of hybrid entrepreneurship.  Such a reexamination is necessary as the proliferation of technologies enable a flexibility to pursue venture opportunities in ways hitherto unexplored \cite{folta2024call}.  Explicitly considering the length of time that entrepreneurs persevere in a hybrid state without exiting their entrepreneurial activity or exiting their wage employment is an opportunity for scholars to expand and clarify the temporal boundary conditions of hybrid entrepreneurship.

Explanations for why entrepreneurs might extend the length of time they remain in a hybrid state include financial motives such as a saturated market for their offering \cite{gras2014risky} and extending time for fundraising efforts \cite{caliendo2020entrepreneurial}. Nonfinancial motives also exist, such as opportunity costs to exit wage employment being too high as can be the case with academic entrepreneurs \cite{hammarstrom2014pursuing, nicolaou2003academic,nicolaou2003social} and the flexibility to pursue additional education or experiences aimed at further entrepreneurship \cite{okolie2024longitudinal}.  

Hybrid entrepreneurship research needs to address temporal differences that are present when studying this phenomenon.  Differentiating between shorter-term and longer-term hybrid entrepreneurship provides a clear framework for such a temporal study.  Presently there are no theoretical distinctions that exists to separate perseverant hybrid entrepreneurs and traditional hybrid entrepreneurs. Temporal distinctions in venture outcomes have been explored in nascent entrepreneurship with less than three years in a venture \cite{lewis2024doing}.  From a practical sense, entrepreneurial activity evaluation that relies on future cash flow analysis generally projects three to five years in the future, beyond which the forecast accuracy diminishes.  The long-term orientation literature accepts the five-year mark as the point where a family firm begins to be viewed as a long-term activity \cite{lebreton2006why, lumpkin2011long}.  Thus, a reasonable threshold for differentiating persistent hybrid entrepreneurship and traditional hybrid entrepreneurship would be expected to exist around the three to five year mark.    To expand scholarly theorizing on hybrid entrepreneurship by explicitly considering a temporal element, we draw upon the long-term orientation perspective. 

\subsection{Self-Determination Theory}
treatus on SDT


\section{Data}
In order to investigate perseverance and aspirations for greater commitment among hybrid entrepreneurs, I sought a sample that would be bound in the following ways.  First, I sought a context where hybrid entrepreneurs would be able to start in a low market and funnel up market to a customer base that is increasingly selective and the relative compensation increases to allow for aspirations for greater commitment (Henley, 2007).  Second, our context must be able to maintain wage employment while engaging in entrepreneurial activities \cite{folta2010hybrid} that cannot be expanded via subcontractors or employees \cite{devries2020explaining} in order to eliminate the possibility of professional management compensating for a lack of full-time entrepreneurial involvement. Third, the context must be ad hoc/at will independent contractors to allow for entrepreneurial decision making \cite{shepherd2022machines,shepherd2015thinking}.  Finally, I sought a context in which hybrid entrepreneurs would be able to persevere indefinitely by virtue of a slow-moving, stable, and saturated market \cite{burtch2018can}. 

A sample that meets the above criteria are sports officials in the United States. Sports officials are appropriate for this study because first, they start at a grassroots level and move up market to more elite, formal, limited and higher paying officiating opportunities (>3,000 USD per contest).  Second, many sports officials maintain wage employment to accommodate the seasonality of officiating. Third, sports officials operate as ad hoc/at will independent contractors with the exception of a select few sports officials that are hired directly with fixed-fee contracts per season or as wage employees rather than independent contractors who are paid per contest.  Such employment situations can only be found for those who officiate some of the very large professional leagues including Major League Baseball (MLB), the National Basketball Association (NBA), the National Football League (NFL), and Major League Soccer (MLS).  Indeed, nearly all full-time officials in the United States, outside the MLB, NBA, NFL, and MLS remain independent contractors given full-time wage employment as an official is incredibly rare (~ 375 employed sports officials total between MLB, NBA, NFL, and MLS which is ~1.6 percent of all sports officials reported by the US Bureau of Labor Statistics (BLS, n.d.)), even for elite officials \cite{lengermann2015adjunct, lewis1999after}.  Indeed, many full-time sports officials, particularly in leagues with shorter seasons such as the MLS, maintain additional employment to supplement their professional officiating compensation.  Regardless of the customer who utilizes a sports official’s services, such work strictly adheres to the US Internal Revenue Service’s (IRS) definition  of independent contractor \cite{irs2024independent}, rather than existing in an employee relationship, thus precluding sports officiating from being considered secondary employment.  Finally, sports officials are in a stable market with regards to the number of customers seeking sports official’s services given sporting competitions see little structural disruption from year-to-year.  
Our dataset is drawn from the 2023 National Association of Sports Officials (NASO) survey of 35,813 sports officials. The survey is administered by NASO on an intermittent basis to their nation-wide membership of United States sports officials representing over 15 sports at levels of competition ranging from youth to international competitions.  The survey respondents included both current and inactive sports officials, though 90.5 percent of respondents considered themselves to be active sports officials (n = 32,411). 


\section{Methods}
 
\subsection{Dependent Variable}
Hybrid Entrepreneurial Persistence. I operationalize hybrid entrepreneurial persistence as the length of time an individual has been an active sports official. This was calculated by subtracting the year respondents stated they started officiating from the year the survey was given (2023).  Our resulting range was 0 to 74 years.  This calculation resulted in individuals with less than a single year of experience to be reported as 0 years.  The respondents that fall into this category are minimal (1.12 percent, n = 401).  

\subsection{Independent Variables}
Hybrid Entrepreneurial Competence. I measure hybrid entrepreneurial competence by an official’s self-reported percentile ranking compared to other officials in their primary sport.  This is consistent with other studies that used self-reported measures given the latent nature of this construct \cite{asante2022entrepreneurial, chen1998does, denoble1999entrepreneurial, pollack2019hybrid}.  Following other studies that have operationalized this measure using a percentile ranking system rather than a multipoint Likert-style scale \cite{kuhnen2018noncognitive}, I felt the ranking system gave more relational context to the saturated market restrictions that would not have been as well-conveyed through a Likert-style scale.  Using the inference of rank within a given market produces a fuller understanding of one’s belief in their ability to complete a task or tasks in a given market.  Coding was from 6 (high) to 1 (low): (6) top 1st – 5th percentile, (5) top 6th – 15th percentile, (4) top 16th – 25th percentile, (3) 26th – 50th percentile, (2) 51st – 75th percentile, and (1) bottom 76th – 100th percentile.  	

Hybrid Entrepreneurial Autonomy. I measure hybrid entrepreneurial autonomy by an official’s claims whether or not they would persist as an official if their non-officiating wages left officiating compensation unneeded.  This is consistent with other studies that have equated financial independence to increased autonomy \cite{howard2024guaranteed}.  Given that autonomy measures lack generalizability and must be tailored to specific situations \cite{lumpkin2009understanding}, our measurement is consistent with latent nature of autonomy in the case of sports officials.  Coding was dichotomous with 1 for yes and 0 for no.  Those who said they were unsure were omitted (n = 2,371).  	

Hybrid Entrepreneurial Relatedness. Given self-determination theory describes relatedness as a need to feel belonging and connectedness \cite{ryan2000self}, I measure hybrid entrepreneurial relatedness as opportunities for situations where socialization can occur within the established officiating structure.  Each respondent reported how the officiating structure within their sport is organized.  Each individual reported whether the officiating structure contained/offered any of the following nine elements: Officiating Coordinators/Assignors, Trainers/Instructors, Assessors, Formal assessment process, Assigned Mentors, Classroom-training sessions, Field-training sessions, Meetings, Formal assessment process, and/or Centralized online portal/site for officials. Formal assessment process and Centralized online portal/site for officials were excluded as neither element requires direct human interaction.  The remainder of the available elements were summed for a total range of 0 to 7, with 7 being the most opportunity for socialization and 0 being no opportunity for socialization. 

\subsection{Control Variables}
Drawing upon the demographic information included in the NASO survey, I utilized a number of controls. Independent contractor is a dichotomous variable determining if an individual operates their hybrid pursuit as an independent contractor (n = 24,289), or as a volunteer or employee (n = 2,147).  Wage Income is a continuous variable that measures the annual income from wage employment only and not officiating activities.  This control is included to account for the possibility of greater opportunity costs promoting hybrid entrepreneurial perseverance \cite{raffiee2014should, tong2020relative}.  Age is a continuous variable that measured the current age of the official used to determine any age-limiting effects on persistence, as older officials are naturally less likely to persevere as long as younger officials.  Number of Children is a continuous variable that controls for the possibility that family obligations limit time spent as an official.  Education level is a continuous variable that ranges from 0 meaning “no degree” to 9 meaning “doctorate degree”.  Current Commitment reflects an official’s current commitment to officiating using the same levels as the dependent variable (aspirations for greater commitment) mentioned earlier. Given perceptions that sports are typically male dominated, I included Gender as a binary variable with responses of Male coded as 1 (n = 23,737) and Female coded as 0 (n = 2,430).  
\subsection{Analysis}
The data were cleansed by removing any respondents that omitted a response for any of the survey questions that were used in our analysis as independent or dependent variables.  The resulting cleansed data (n= 16,700) was then transformed into z-scores for ease of interpretation.  I tested our hypotheses using an OLS regression predicting how the length of time hybrid entrepreneurs have spent in their entrepreneurial activity is related to their aspirations for greater commitment to that activity.

\LaTeX{} is great at typesetting mathematics. Let $X_1, X_2, \ldots, X_n$ be a sequence of independent and identically distributed random variables with $\text{E}[X_i] = \mu$ and $\text{Var}[X_i] = \sigma^2 < \infty$, and let
\[S_n = \frac{X_1 + X_2 + \cdots + X_n}{n}
      = \frac{1}{n}\sum_{i}^{n} X_i\]
denote their mean. Then as $n$ approaches infinity, the random variables $\sqrt{n}(S_n - \mu)$ converge in distribution to a normal $\mathcal{N}(0, \sigma^2)$.


\section{Findings}
Variable means and correlations reported in Table 2. Table 3 presents the results related to Hypotheses 1, 2 and 3 using OLS regression. Model 1 is a controls-only baseline model.  Model 2 contains the number of years an individual has been a sports official, representing hypothesis 1.  Models 3 and 4 contain results for an individual’s self-rank for hypothesis 2, and the robustness test of self-confidence explained later.  Models 5 and 6 represent the inclusion of self-rank as moderation between hybrid entrepreneurial perseverance and aspiration for greater commitment by entrepreneurial self-efficacy in hypothesis 3 and the interaction terms where measures of self-rank and self-confidence are included as robustness checks.  
-----
Insert Tables 2 and 3 here
-----
 Our analysis reveals that the years an official spends officiating is positively associated with aspirations for greater time commitment (β = 0.061, p < 0.001), supporting Hypothesis 1. Consistent with self-efficacy theory, perceived self-rank was positively related to future commitment aspirations (β = 0.059, p < 0.001) supporting hypothesis 2. Finally, I found that self-rank weakened the relationship between years spent officiating (hybrid entrepreneurial perseverance) and aspirations for greater commitment (β = -0.012, p < 0.01).  While statistically significant, this result is contrary to theoretical predictions and opposite of what I hypothesized, though Hypothesis 3 is supported. I show this moderation graphically in Figure 2 as an interaction plot.
-----
Insert Figure 2 here
-----
Robustness Checks
	Self Confidence. An entrepreneur’s perceived self-rank is a strong proxy for entrepreneurial self-efficacy.  Yet, given our moderating variable is both self-reported and ordinal in nature I ran a robustness test (see table 2) on a second measure of self-efficacy utilizing another question from the NASO survey that asked, “How would you describe your level of confidence/certainty regarding decisions while officiating each sport?” Responses were given with 1 as the highest possible and 10 being the lowest possible (we reverse scored these for our analysis).  I thus confirm that our results (β = 0.066, p < 0.001) are consistent with self-rank as are the interaction between confidence and years spent as official ($\beta$ = -0.013, p < 0.01). 
As the weakening of the interaction effect in Hypothesis 3 was contrary to theory, I conducted a robustness check using an ordered probit model as an alternate model specification.  An ordered probit model is used to analyze ordinal categorical dependent variables \cite{agresti2012categorical}.  As our dependent variable levels could be interpreted as ordinal and categorical, I conducted an ordered probit model analysis and subsequently found consistent results to those from the OLS regression (see Table 4).



\subsection{Post Hoc Analysis}
Adding in a question if people felt their home life was damage, there was a significance relationship.  If a person’s home life is damaged by officiating, they either retreat to officiating for about 3.6 more years or they attach additional time to officiating to make the investment worth it.  Looking into the home relationships around persistent hybrid entrepreneurs would be a meaningful contribution for future research.


\section{Conclusions}

Rehash my contributions

\subsection{Implications for Theory}
Expand on hybrid entrepreneurship by bringing in persistence perspective and how self-determination theory can explain this.


\subsection{Implications for Practice}
expand on implications for entertainment industry

\subsection{SUGGESTIONS FOR FUTURE RESEARCH}
As autonomy measures typically lack generalizability \cite{lumpkin2009understanding}, further research would benefit from differentiation in different hybrid entrepreneurial contexts, especially instances of solopreneurship. 

\subsection{How to include Figures}

First you have to upload the image file from your computer using the upload link in the file-tree menu. Then use the includegraphics command to include it in your document. Use the figure environment and the caption command to add a number and a caption to your figure. See the code for Figure \ref{fig:frog} in this section for an example.

Note that your figure will automatically be placed in the most appropriate place for it, given the surrounding text and taking into account other figures or tables that may be close by. You can find out more about adding images to your documents in this help article on \href{https://www.overleaf.com/learn/how-to/Including_images_on_Overleaf}{including images on Overleaf}.

\begin{figure}
\centering
\includegraphics[width=0.25\linewidth]{frog.jpg}
\caption{\label{fig:frog}This frog was uploaded via the file-tree menu.}
\end{figure}

\subsection{How to add Tables}

Use the table and tabular environments for basic tables --- see Table~\ref{tab:widgets}, for example. For more information, please see this help article on \href{https://www.overleaf.com/learn/latex/tables}{tables}. 

\begin{table}
\centering
\begin{tabular}{l|r}
Item & Quantity \\\hline
Widgets & 42 \\
Gadgets & 13
\end{tabular}
\caption{\label{tab:widgets}An example table.}
\end{table}

\subsection{How to change the margins and paper size}

Usually the template you're using will have the page margins and paper size set correctly for that use-case. For example, if you're using a journal article template provided by the journal publisher, that template will be formatted according to their requirements. In these cases, it's best not to alter the margins directly.

If however you're using a more general template, such as this one, and would like to alter the margins, a common way to do so is via the geometry package. You can find the geometry package loaded in the preamble at the top of this example file, and if you'd like to learn more about how to adjust the settings, please visit this help article on \href{https://www.overleaf.com/learn/latex/page_size_and_margins}{page size and margins}.

\bibliographystyle{abbrv}
\bibliography{PS11_Canova}

\end{document}